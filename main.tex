\documentclass[letterpaper]{article}
\usepackage{plainresume}
\usepackage{ifthen}

\begin{document}

\title{Aravind Sundaresan}
\author{Aravind Sundaresan}

% select home or office address

\def\myaddress{office}
%\def\myaddress{home}

\ifthenelse{\equal{\myaddress}{office}}{%
  \address{EK252, SRI International 
  333 Ravenswood Ave, 
  Menlo Park, CA 94025.}
}{%
  \address{407 Acalanes Dr, Apt 13, Sunnyvale, CA 94086}
}

\phone{(240) 678-9012}
\fax{(650) 859-3735}
\email{aravind@ai.sri.com} 
\homepage{\href{http://www.ai.sri.com/~aravind/}{http://www.ai.sri.com/$\sim$aravind/}}

\maketitle

%\section*{Objective}
%\begin{list1}
%  \item 
%	I am looking for a full-time position in applied research in the fields of
%	\emph{Computer Vision}, \emph{Image Processing}, and \emph{Pattern
%	Recognition}. I am interested in building vision systems for solving and
%	aiding real world problems. 
%\end{list1}



\section*{Education}
\begin{list1}
 \item
  \begin{tabular1bold}
   M.S. and Ph.D., Electrical Engineering & Aug, 2001 -- Aug, 2007 \\
   University of Maryland, College Park, MD. & \\
  \end{tabular1bold}

 \begin{list2}
  \item Thesis : ``Towards markerless motion capture: Model estimation, pose
	initialisation and tracking.''
%  \item GPA: 3.73, \mbox{ } 
%		Major: Communications and Signal Processing, \mbox{ } 
%		Minor: Controls Engineering
%  \item Selected Courses: Random Processes, Advanced Digital Signal
%   Processing, Estimation and Detection Theory, Information Theory, System Theory,
%   Image Processing, Statistical and Neural Recognition of Patterns, Image
%   Understanding, Computer Processing of Pictorial Information, Scientific
%   Computing. 
 \end{list2}

 \item
  \begin{tabular1bold}
   B.E. (Hons.), Electrical and Electronics Engineering & Aug, 1996 -- May, 2000 \\
   Birla Institute of Technology and Science, Pilani, Rajasthan, India. & \\
  \end{tabular1bold}

  \begin{list2}
  \item 
	    %GPA: 9.11/10.00, \mbox{ }
        Major: Communications and Signal Processing, \mbox{ }
		Minor: Computer Engineering
  \end{list2}

\end{list1}



\section*{Academic and Professional Experience}
\begin{list1}
 \item
  \begin{tabular1bold}
	% Complete Address CA 94041
   Computer Scientist, SRI International, Menlo Park, CA. & Aug, 2007 -- Present\\
  \end{tabular1bold}

  \begin{list2}
   \item Built a real-time multi-sensor based people tracking system as part of
	 the Sentient project.
   \item Built a system that includes comprehensive localization, mapping, and
	 planning techniques for a mobile robot to operate autonomously in complex
	 3D indoor and outdoor environments in the ``Leaving Flatland'' project.
   \item Designed and implemented a Path detection algorithm using both texture
	 and 3D information for navigation in an autonomous robot as part of the
	 LAGR project.
  \end{list2}
  
 \item
  \begin{tabular1bold}
   Research Assistant, Centre for Automation Research, College Park, MD. 
   & Aug, 2001 -- Aug, 2007\\
  \end{tabular1bold}

  \begin{list2}
   \item Designed and implemented a markerless motion capture system as part of the NSF-funded project {\em ``New Technology for Capture, Analysis and Visualisation of Human Movement Using Distributed Cameras using an articulated body model and multiple cameras''}.  Project is a collaboration with the Biomotion Laboratory at Stanford University and the Media Research Laboratory at New York University. 

  \item Designed \emph{Hydra}\footnote{\href{http://www.cfar.umd.edu/users/aravinds/research/hydra.html}{http://www.cfar.umd.edu/users/aravinds/research/hydra.html}}, a portable and flexible synchronised
	multi-camera capture facility at the University of Maryland for human motion
	analysis. The scalable system currently includes ten Firewire cameras
	connected to two headless nodes running Debian Linux and synchronised using
	a custom circuit. 

  \item Wrote a suite of programs using the open-source \path{libdc1394}
	library to capture, calibrate and perform synchronised capture using
	firewire cameras. Contributed PixeLINK camera-specific code to the
	\path{libdc1394} project.

  \item Experienced in multiple camera capture and calibration in several
	environments such as \emph{Hydra}, the KECK
	laboratory\footnote{\href{http://www.cfar.umd.edu/users/aravinds/research/keck.html}{http://www.cfar.umd.edu/users/aravinds/research/keck.html}} (24 cameras) at the University of Maryland, and at the Honda Research Institute (8 PULNiX cameras for realtime capture). 

  \item Part of a team that worked on DARPA-funded project titled {\em``Video
	Verification and Identification''}. Designed and implemented
	in the C$^{++}$ language, an HMM based algorithm for identification of humans 
	from gait video sequences. 

  \end{list2}
   
 \item
  \begin{tabular1bold}
	% Complete Address CA 94041
   Intern, Honda Research Institute Inc., Mountain View, CA. & Jun, 2004 -- Aug, 2004\\
  \end{tabular1bold}

  \begin{list2}
   \item Built a real-time marker-based motion capture system as part of a
	 project to redirect human motion to a generic robot. Images were captured
	 from eight PULNiX Cameras attached to two servers running Mandrake Linux.
	 Wrote a socket-based program to control, calibrate and capture from 
	 cameras.
  \end{list2}
  
  \item
   \begin{tabular1bold}
	Software Engineer, Sasken Communication Technologies Ltd., Bangalore.
	& Jul, 2000 -- Jul, 2001\\
   \end{tabular1bold}

   \begin{list2}
	\item Implemented the ISO MPEG-4 and ITU-T H.263 standards codec on
	 SHARP's DDMP (Data Driven Media Processor), and Matsushita Electronics' MMP
	 (Mobile Multimedia Processor). 

	\item Implemented the ITU-T H.223 standard (Annexure C) for Error Correction
	 in a Mobile Environment in the C language. 
   \end{list2}

  \item
   \begin{tabular1bold}
	Intern, Motorola India Electronics Limited, Bangalore. & Jan, 2000
	-- Jun, 2000\\
   \end{tabular1bold}

   \begin{list2}
	\item Implemented the ITU-T video codec standards H.261 and
	 H.263 in the C Language and Pentium MMX Assembly language. 
   \end{list2}

\end{list1}

\section*{Selected Publications}
\begin{enumerate1}
  \item
	\begin{enumerate2}
	  \item  A. Sundaresan and R. Chellappa, ``Multi-camera tracking of Articulated
		Human Motion'', under revision for {\em IEEE Transactions of Image
		Processing}.

	  \item
		M.~R. Blas, M.~Agrawal, A.~Sundaresan, and K.~Konolige.
		``Fast color/texture segmentation for outdoor robots.''
		In {\em IEEE International Conference on Intelligent Robots and
		Systems}, Nice, France, October 2008.

	  \item
		R.~B. Rusu, A.~Sundaresan, B.~Morisset, M.~Agrawal, and M.~Beetz.
		{``Leaving Flatland: Realtime 3D Stereo Semantic Reconstruction''}.
		In {\em Proceedings of the International Conference on Intelligent
		Robotics and Applications}, Wuhan, China, October 2008.

	  \item  A. Sundaresan and R. Chellappa, ``Model driven segmentation and
		registration of articulating humans in Laplacian Eigenspace'', 
		{\em IEEE Transactions of Pattern Analysis and Machine Intelligence}.
		30(10), October 2008.
		\label{pub:pami06}

	  \item A. Sundaresan, R. Chellappa, ``Markerless Motion Capture using Multiple
		Cameras'', {\em Computer Vision for Interactive and Intelligent
		Environments, (Eds. C. Jaynes and R. Collins)}, IEEE Press, 2006.

	  \item A. Kale, A. Sundaresan, A. RoyChowdhury, and R. Chellappa, ``Gait-Based
		Human Identification From A Monocular Video Sequence'', {\em Handbook on
		Pattern Recognition and Computer Vision (Eds. C. H. Cheng and P. S. P.
		Wang)}, 3rd Ed, World Scientific Publishing Company Pvt. Ltd., 2005. 

	  \item A. Kale, A. Sundaresan, A. N. Rajagopalan, N. Cuntoor, A. RoyChowdhury,
		V. Kruger, R. Chellappa, ``Identification of Humans Using Gait'', {\em IEEE
		Transactions on Image Processing}, September 2004. 

	  \item  A. Sundaresan and R. Chellappa, ``Segmentation and Probabilistic
		Registration of Articulated Body Models'', {\em International Conference on
		Pattern Recognition}, Hong Kong, 2006. [Best Student Paper Award]
		\label{pub:icpr06}

	  \item  A. Sundaresan and R. Chellappa, ``Multi-camera Tracking of Articulated
		Human Motion using Motion and Shape Cues'', {\em Asian Conference on
		Computer Vision}, Hyderabad, January 2006. 

\comment{ %%%% BEGIN COMMENT %%%%
	  \item  A. Sundaresan, A. RoyChowdhury, and R. Chellappa, ``Multiple View
		Tracking of Human Motion Modelled by Kinematic Chains'', {\em International
		Conference on Image Processing}, Singapore, October 2004. 
		} %%%%  END COMMENT  %%%%

	  \item A. Sundaresan, A. Roy Chowdhury, R. Chellappa. ``A Hidden Markov
		Model for Recognising Humans from Gait'', {\em International Conference
		on Image Processing}, Barcelona, September 2003.

	  \item A. Sundaresan and R. Chellappa, "Acquisition of Articulated Human Body
		Models using Multiple Cameras", {\em IV Conference on Articulated Motion and
		Deformable Objects}, Andratx, Mallorca, Spain, 2006. 
		\label{pub:amdo06}

\comment{ %%%% BEGIN COMMENT %%%%
	  \item L. M\"{u}ndermann, S. Corazza, A. M. Chaudhari, T. P. Andriacchi, A.
		Sundaresan, and R. Chellappa, "Measuring human movement for biomechanical
		applications using markerless motion capture", {\em IS\&T / SPIE 18th Annual
		Symposium: Electronic Imaging}, San Jose, California, 2006. 

	  \item A. Sundaresan, A. Roy Chowdhury, R. Chellappa. ``3D Modelling of Human
		Motion Using Kinematic Chains and Multiple Cameras for Tracking'', 
		{\em Eighth International Symposium on the 3-D Analysis of Human
		Movement}, Tampa, March 2004.
		} %%%%  END COMMENT  %%%%

	\end{enumerate2}
\end{enumerate1}

\section*{Selected Talks}
\begin{list1}
  \item
	\begin{list2}
	  \item ``Segmentation in Laplacian Eigenspace for Model and Pose
		Estimation''. Biomotion Laboratory, Stanford University, Palo Alto, CA.
	  \item ``Towards markerless motion capture using Computer Vision''. Yahoo Research, Bangalore, India.
	\end{list2}
\end{list1}

\section*{Patents}
\begin{list1}
  \item
	\begin{list2}
	  \item Patent application filed (Application No. 60/865,207)
		with Prof. Rama Chellappa for ``Markerless Motion Capture'' using the
		algorithms presented in [\ref{pub:pami06}], [\ref{pub:icpr06}], and
		[\ref{pub:amdo06}] under \textsc{Selected Publications}.
	\end{list2}
\end{list1}

%\section*{Selected course projects}
%\begin{list1}
% \item
%  \begin{list2}
%   \item JPEG-like Codec and MPEG4-like Codec with shot segmentation of video streams using wipe detection techniques. 
%
%   \item Edge Detection using Marr-Hildreth and Haralick edge detectors.
%
%   \item Texture Segmentation using Markov Random Fields.
%
%   \item Structure from Motion from multiple frames using Iterated Extended Kalman
%	Filter and Lucas-Kanade factorisation.
%
%   \item Mosaicking of images obtained from rotating camera.
%
%   \item Motion Segmentation using calibrated stereo rig.
%
%   \item Parametric Speech Coding using Linear Prediction.
%  \end{list2}
%\end{list1}

\section*{Computer Skills} 
\begin{list1}
 \item
  \begin{list2}

   \item Mathematical Packages: MATLAB, Octave.

   \item Languages: C, C$^{++}$, UNIX Shell Scripting using Bash, Perl, Assembly
	 Language (Pentium, DSP).

   \item Operating Systems: Linux (RedHat, Gentoo, Ubuntu, Debian), UNIX
	(SunOS), Windows.

   \item Miscellaneous: \LaTeX, HTML, CSS.
  \end{list2}
\end{list1}


\section*{Honours and Awards}
\begin{list1}
  \item
	\begin{list2}
	  \item Best Student Paper award at the biennial International Conference on
		Pattern Recognition 2006 in Computer Vision and Image Analysis for 
		\emph{``Segmentation and Probabilistic Registration of
		Articulated Body Models''}.
	  \item University of Maryland \emph{``Outstanding Invention of 2006''}
		award for \emph{``Markerless Motion Capture''}.
	  \item All India Rank 2 in the Indian School Certificate
		Examination, 1996.
	  \item All India Rank 1021 in the Indian Institute of Technology- Joint
		Entrance Examination, 1996.

	\end{list2}
\end{list1}

\comment{
\section*{References} 
\begin{list1}
 \item
  \begin{list2}
   \item Rama Chellappa\\
	 Minta Martin Professor of Engineering,  \\
	 Director, Center for Automation Research \\
	 Department of Electrical and Computer Engineering, \\
	 University of Maryland, College Park, MD 20742. \\
	 Email: rama@cfar.umd.edu  Phone: +1 301 405 3656

   \item Amit RoyChowdhury \\
	 Asst Professor, Department of Electrical Engineering, \\
	 University of California, Riverside, CA 92521. \\
	 Email: amitrc@ee.ucr.edu   Phone: +1 951 827 7886

   \item James Davis, 
	 Asst Professor, Computer Science, 
	 University of California, Santa Cruz, CA 95064. \\
	 Email: davis@cs.ucsc.edu   Phone: +1.831.459.1841

   \item Min Wu\\
	 Associate Professor, Department of Electrical and Computer Engineering, \\
	 University of Maryland, College Park, MD 20742. \\
	 Email: minwu@eng.umd.edu  Phone: +1 301 405 0401

   \item Dr. Hector Gonzalez-Banos, Honda Research Institute Inc, Mountain View,
	California.

  \end{list2}
\end{list1}

}

\end{document}

