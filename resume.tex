\documentclass[letterpaper]{scrartcl}
\usepackage{nicefrac}
\usepackage{plainresume3}
\usepackage{changepage} % used for adjustwidth. Not required if those are removed
\pagestyle{plain}
%\author{Dan Luu}

\begin{document}
%\title{Dan Luu}
\title{\vspace{-6ex}\tt{dan.luu@gmail.com}}
\date{\vspace{-10ex}}
\author{}
%\maketitle

\email{danluu@gmail.com} 
%\footnotesize
%\setstretch{.9}
%
%-------------------------------------------------------------------------------
%

%\vspace{-3pt}

\mag 1000

%
%-------------------------------------------------------------------------------
%

%\vspace{-3pt}
\section*{Experience}
%\vspace{-3pt}

\begin{list1}

\item \begin{tabular1bold} Senior Hardware/Software Engineer, Google; Madison, WI & 2013 -- 2014 \end{tabular1bold}

  \begin{list2}
  \item Exploratory hardware/software co-design; details confidential \hfill \emph{---} %FIXME: mixing tenses
  \end{list2}

\item \begin{tabular1bold} Sabbatical, Hacker School; New York, NY & Spring 2013 \end{tabular1bold}

  \begin{list2}
  \item Projects include channels and coroutines in C and a BitTorrent client in Scala.
  \item See https://github.com/danluu/ and http://danluu.com for more.
  \end{list2}
\item \begin{tabular1bold} Member of Technical Staff, Centaur Technology (acquired by VIA); Austin, TX & 2005 -- 2013 \end{tabular1bold}

  %\begin{adjustwidth}{1.5em}{1.5em}
  \begin{list2}
  \item Here's one particular six-month project (adding an ARM front-end to our x86):
    \begin{list3}
    \item Helped reverse engineer the ARMv7 ISA (this was pre-AArch64)
    \item Created architectural simulator and got Android running on it\hfill \emph{C}
    \item Implemented \nicefrac{1}{2} of the translator, and wrote associated microcode \hfill \emph{Internal templating language}
    \item Created test generator that found 90\% of the first 1000 bugs on the project \hfill \emph{F\#}
    \item Result was a circa 2010 ARMv7 processor with better performance than any current ARM processor
    \end{list3}
  \item Other roles (often brief) included formal verification, adding fault tolerance to a distributed system, post-silicon debug, test tooling, etc.
    
  \end{list2}
  %\end{adjustwidth}

\item \begin{tabular1bold} Research Assistant, Ultrafast Optics and Fiber Communications Lab; Lafayette, IN & 2003 -- 2005 \end{tabular1bold}

  \begin{list2}
  \item Sped up parallel (256 wavelength) polarimeter by 40x, from 50 Hz to 2 kHz \hfill \emph{MATLAB and C}
  \item Designed and built Fourier transform spectroscopy interferometer \hfill \emph{MATLAB and C}
  \end{list2}

\item \begin{tabular1bold} Intern, IBM; Austin, TX & Summer 2003 \end{tabular1bold}

  \begin{list2}
  \item Semi-formal / constrained random POWER6 completion unit functional verification \hfill \emph{VHDL}
  \end{list2}

\item \begin{tabular1bold} Intern, Micron Technology; Boise, ID & Summer 2002 \end{tabular1bold}

  \begin{list2}
  \item Engineering hipster: worked on flash before it was cool \hfill \emph{Perl}
  \end{list2}
\item \begin{tabular1bold} Research Assistant, Spatial Systems Research Laboratory; Madison, WI & 2001 \end{tabular1bold}

  \begin{list2}
  \item Studied tilings and related combinatorial models, e.g., alternating sign matricies and square ice
  \end{list2}
\end{list1}

\section*{Education}
%\vspace{-3pt}
\begin{list1}
  %	\item
  %	\begin{list2}
  %	\item Selected Graduate Courses: Computer Architecture, Interconnect Modeling and Optimization, VLSI Design, Digital Logic Synthesis Algorithms, Computational and Statistical Learning Theory, Empirical Methods in Engineering, Matrix Theory, Error-Correcting Codes, Adv. Math for Engineers, Theory of Differential Equations, Algorithms
  %	\item Selected Upper Division Undergraduate Courses: CMOS VLSI Design, Testing and Design for Testability, Digital Systems Design and Synthesis, Databases, Combinatorics
  %	\end{list2}
\item
  \begin{tabular1bold}BS Math \& CMPE (Wisconsin, '00-'03), MS EE (Purdue, '03-'05)\end{tabular1bold}
\end{list1}

\section*{Non-work Projects}
%\vspace{-3pt}
\begin{list1}
\item
  \begin{list2}
  \item See https://github.com/danluu/ and http://danluu.com for an exhaustive list
    \begin{list3}
    \item Sega system on Xilinx Vertex FPGA; translated Z80 instructions into RISC $\mu$ops \hfill \emph{Verilog and VHDL}
    \item Experiments with a randomized algorithm as cache eviction policy \hfill \emph{Dinero IV and SPEC}
    \item A fuzzer written in an hour that found ~20 bugs in the Julia compiler and base libraries \hfill \emph{Julia}
    \item Formal verification of a secure hypervisor model \hfill \emph{ACL2}
    \end{list3}


  \end{list2}
\end{list1}


%
%-------------------------------------------------------------------------------
%
%\vspace{-3pt}
%% \section*{Honors and Awards}
%% %\vspace{-3pt}
%% \begin{list1}
%% \item
%%   \begin{list2}
%%   \item MCD Fellowship \hfill 2009 - 2010
%%   \item Burton D. Morgan Entrepreneurship Competition Semi-Finalist \hfill 2005
%%   \item David Ross Fellowship (five years of guaranteed funding) \hfill 2003 - 2005
%%   \item SRC undergraduate research grant \hfill 2001 - 2003
%%   \item Dean's List \hfill 2001 - 2003
%%   \item VIGRE undergraduate research funding \hfill 2001
%%   \item AP Scholar with distinction \hfill 2000
%%   \end{list2}
%% \end{list1}

%\begin{comment}
%\vspace{-3pt}
\section*{Miscellaneous}
%\vspace{-3pt}
\begin{list1}
\item
  \begin{list2}
  \item Languages: English mother tongue. Once-fluent Vietnamese. Once-functional (now moribund) Japanese and French. Willing (and eager) to learn any language
    %\item \coltabone{PL}{Once-fluent (now rusty) Ruby. Once-functional (now moribund) x86, ACL2, C, C++, Verilog, and VHDL. Willing (and eager to) learn any language}
    %		\item Activities: Officer and organizer for Students in Software Engineering; member of ARG climbing team, UT climbing team, and occasional antendee of UT sciences Toastmasters club
  \item Work Authorization: U.S. Citizen
  \end{list2}
\end{list1}
%\end{comment}
\end{document}
